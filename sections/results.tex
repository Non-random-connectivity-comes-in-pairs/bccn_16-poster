
\section*{Results}

Define a \textbf{random network model} in which node-to-node connection probabilities themselves follow a random distribution; some pairs of neurons have a higher chance to be connected than others.


Consider a random network model of $N$ neurons in which a connection from node $i$ to node $j$ exists with probability $P_{ij}$.
%
Here, the $P_{ij}$, with $i,j = 1,\dots,N$ and $i \neq j$, are identically distributed random variables in $[0,1]$, yielding a probability of connection for each ordered pair of nodes in the network.

\textbf{\textcolor{gblue}{Example}}: Network in which connection probability is either high or low, depending on whether the target neuron share the same feature (e.g. orientation tuning) or not.

\begin{center}
  \rule{19cm}{2pt}
\end{center}

The overall connection probability $\mu$, that is the chance to have a connection from a random node $i$ to another node $j$, is
\begin{align}
\mu = \E(P_{ij}).
\end{align}
For example, if the $P_{ij}$ have a probability density function $f$ with essential support in $[0,1]$, we can compute the connection fraction as
\begin{align}
  \mu = \int_0^1 x f(x)\,dx.
\end{align}
The probability for a random pair of nodes $i$ and $j$ to be connected reciprocally is then
\begin{align}
P_{\mathrm{rec}} = \E(P_{ij} P_{ji}).
\end{align}

\begin{center}
  \rule{19cm}{2pt}
\end{center}

The relative occurrence $\varrho$ of these reciprocally connected pairs within the network compares $P_{\mathrm{rec}}$ with the occurrence of bidirectionally connected pairs in an Erd\H{o}s-R\'{e}nyi graph, in which each unidirectional connection is equally likely to occur with probability $\mu$. Formally,
\begin{align}
  \varrho = \frac{P_{\mathrm{rec}}}{\mu^2} = \frac{\E\left(P_{ij}P_{ji}\right)}{{\E\left(P_{ij}\right)}^2}.
\end{align}

If connection probabilities are \textbf{symmetric in pairs}, $P_{ij} = P_{ji}$, then
\begin{align}
\varrho = \frac{\E(P_{ij}^2)}{{\E\left(P_{ij}\right)}^2}.
\end{align}.

Since $x \mapsto x^2$ is a strictly convex function, \textbf{Jensen's inequality} \cite{Jensen1906} yields
\begin{align}
\E(P_{ij}^2) \geq \E(P_{ij})^2. \label{eq:jensen}
\end{align}
Equality holds in \eqref{eq:jensen} if and only if $P_{ij}$ is a constant. Thus \textit{any} non-degenerate distribution of connection probabilities \textit{necessarily} induces an overrepresentation of bidirectional connections in the network, $\varrho > 1$.

\begin{center}
  \rule{19cm}{2pt}
\end{center}

\textcolor{gblue}{\textbf{Summary}}: In a network where some pairs are more likely connected than others, the count of expected reciprocally connected pairs is strictly underestimated by the statistics of an Erd\H{o}s-R\'{e}nyi graph with same overall connection probability $\E(P_{ij}) = \mu$.

